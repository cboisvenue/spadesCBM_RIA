% Options for packages loaded elsewhere
\PassOptionsToPackage{unicode}{hyperref}
\PassOptionsToPackage{hyphens}{url}
%
\documentclass[
]{article}
\usepackage{amsmath,amssymb}
\usepackage{lmodern}
\usepackage{ifxetex,ifluatex}
\ifnum 0\ifxetex 1\fi\ifluatex 1\fi=0 % if pdftex
  \usepackage[T1]{fontenc}
  \usepackage[utf8]{inputenc}
  \usepackage{textcomp} % provide euro and other symbols
\else % if luatex or xetex
  \usepackage{unicode-math}
  \defaultfontfeatures{Scale=MatchLowercase}
  \defaultfontfeatures[\rmfamily]{Ligatures=TeX,Scale=1}
\fi
% Use upquote if available, for straight quotes in verbatim environments
\IfFileExists{upquote.sty}{\usepackage{upquote}}{}
\IfFileExists{microtype.sty}{% use microtype if available
  \usepackage[]{microtype}
  \UseMicrotypeSet[protrusion]{basicmath} % disable protrusion for tt fonts
}{}
\makeatletter
\@ifundefined{KOMAClassName}{% if non-KOMA class
  \IfFileExists{parskip.sty}{%
    \usepackage{parskip}
  }{% else
    \setlength{\parindent}{0pt}
    \setlength{\parskip}{6pt plus 2pt minus 1pt}}
}{% if KOMA class
  \KOMAoptions{parskip=half}}
\makeatother
\usepackage{xcolor}
\IfFileExists{xurl.sty}{\usepackage{xurl}}{} % add URL line breaks if available
\IfFileExists{bookmark.sty}{\usepackage{bookmark}}{\usepackage{hyperref}}
\hypersetup{
  pdftitle={Appendix 1 The Carbon Budget Model and simulations},
  hidelinks,
  pdfcreator={LaTeX via pandoc}}
\urlstyle{same} % disable monospaced font for URLs
\usepackage[margin=1in]{geometry}
\usepackage{color}
\usepackage{fancyvrb}
\newcommand{\VerbBar}{|}
\newcommand{\VERB}{\Verb[commandchars=\\\{\}]}
\DefineVerbatimEnvironment{Highlighting}{Verbatim}{commandchars=\\\{\}}
% Add ',fontsize=\small' for more characters per line
\usepackage{framed}
\definecolor{shadecolor}{RGB}{248,248,248}
\newenvironment{Shaded}{\begin{snugshade}}{\end{snugshade}}
\newcommand{\AlertTok}[1]{\textcolor[rgb]{0.94,0.16,0.16}{#1}}
\newcommand{\AnnotationTok}[1]{\textcolor[rgb]{0.56,0.35,0.01}{\textbf{\textit{#1}}}}
\newcommand{\AttributeTok}[1]{\textcolor[rgb]{0.77,0.63,0.00}{#1}}
\newcommand{\BaseNTok}[1]{\textcolor[rgb]{0.00,0.00,0.81}{#1}}
\newcommand{\BuiltInTok}[1]{#1}
\newcommand{\CharTok}[1]{\textcolor[rgb]{0.31,0.60,0.02}{#1}}
\newcommand{\CommentTok}[1]{\textcolor[rgb]{0.56,0.35,0.01}{\textit{#1}}}
\newcommand{\CommentVarTok}[1]{\textcolor[rgb]{0.56,0.35,0.01}{\textbf{\textit{#1}}}}
\newcommand{\ConstantTok}[1]{\textcolor[rgb]{0.00,0.00,0.00}{#1}}
\newcommand{\ControlFlowTok}[1]{\textcolor[rgb]{0.13,0.29,0.53}{\textbf{#1}}}
\newcommand{\DataTypeTok}[1]{\textcolor[rgb]{0.13,0.29,0.53}{#1}}
\newcommand{\DecValTok}[1]{\textcolor[rgb]{0.00,0.00,0.81}{#1}}
\newcommand{\DocumentationTok}[1]{\textcolor[rgb]{0.56,0.35,0.01}{\textbf{\textit{#1}}}}
\newcommand{\ErrorTok}[1]{\textcolor[rgb]{0.64,0.00,0.00}{\textbf{#1}}}
\newcommand{\ExtensionTok}[1]{#1}
\newcommand{\FloatTok}[1]{\textcolor[rgb]{0.00,0.00,0.81}{#1}}
\newcommand{\FunctionTok}[1]{\textcolor[rgb]{0.00,0.00,0.00}{#1}}
\newcommand{\ImportTok}[1]{#1}
\newcommand{\InformationTok}[1]{\textcolor[rgb]{0.56,0.35,0.01}{\textbf{\textit{#1}}}}
\newcommand{\KeywordTok}[1]{\textcolor[rgb]{0.13,0.29,0.53}{\textbf{#1}}}
\newcommand{\NormalTok}[1]{#1}
\newcommand{\OperatorTok}[1]{\textcolor[rgb]{0.81,0.36,0.00}{\textbf{#1}}}
\newcommand{\OtherTok}[1]{\textcolor[rgb]{0.56,0.35,0.01}{#1}}
\newcommand{\PreprocessorTok}[1]{\textcolor[rgb]{0.56,0.35,0.01}{\textit{#1}}}
\newcommand{\RegionMarkerTok}[1]{#1}
\newcommand{\SpecialCharTok}[1]{\textcolor[rgb]{0.00,0.00,0.00}{#1}}
\newcommand{\SpecialStringTok}[1]{\textcolor[rgb]{0.31,0.60,0.02}{#1}}
\newcommand{\StringTok}[1]{\textcolor[rgb]{0.31,0.60,0.02}{#1}}
\newcommand{\VariableTok}[1]{\textcolor[rgb]{0.00,0.00,0.00}{#1}}
\newcommand{\VerbatimStringTok}[1]{\textcolor[rgb]{0.31,0.60,0.02}{#1}}
\newcommand{\WarningTok}[1]{\textcolor[rgb]{0.56,0.35,0.01}{\textbf{\textit{#1}}}}
\usepackage{graphicx}
\makeatletter
\def\maxwidth{\ifdim\Gin@nat@width>\linewidth\linewidth\else\Gin@nat@width\fi}
\def\maxheight{\ifdim\Gin@nat@height>\textheight\textheight\else\Gin@nat@height\fi}
\makeatother
% Scale images if necessary, so that they will not overflow the page
% margins by default, and it is still possible to overwrite the defaults
% using explicit options in \includegraphics[width, height, ...]{}
\setkeys{Gin}{width=\maxwidth,height=\maxheight,keepaspectratio}
% Set default figure placement to htbp
\makeatletter
\def\fps@figure{htbp}
\makeatother
\setlength{\emergencystretch}{3em} % prevent overfull lines
\providecommand{\tightlist}{%
  \setlength{\itemsep}{0pt}\setlength{\parskip}{0pt}}
\setcounter{secnumdepth}{-\maxdimen} % remove section numbering
\ifluatex
  \usepackage{selnolig}  % disable illegal ligatures
\fi

\title{Appendix 1 The Carbon Budget Model and simulations}
\author{}
\date{\vspace{-2.5em}}

\begin{document}
\maketitle

\hypertarget{overview}{%
\section{Overview}\label{overview}}

Our study links three existing simulation models to explore their
capacity to support forest carbon (C) management decisions using the
current metrics of sustainable forest management practices (levels of
\(m3/ha\)). Here, we describe the scripts used in our simulations. These
scripts are R-based and are repeatable and transparent. The central
model of our simulations is the Carbon Budget Model of the Canadian
Forest Sector (CBM - Kurz et al., 2009). We implemented faithfully the
logic, pools structure, equations, and default assumptions of the CBM
into the meta R-package, \texttt{SpaDES} (Spatial Discrete Event
Simulation - McIntire et al., 2019). This resulted in a spatialized
version of the CBM used in simulations in conjunction with landscape
modifications from a harvesting scheduler ws3 (Appendix 2) and a fire
model (Appendix 3). The yearly processing of the entire landscape
permits the link to our harvesting scheduling and fire model, which are
also in \texttt{SpaDES}, via yearly landscape modifications. Linking
these three models permits the assessment of carbon implication of
harvesting and fire disturbances on our study area (see Figure 1). All
scripts are R-based, providing a parameter and data handling platform,
and a clear understanding of the model structure and parameters to any
moderately R-proficient scientist.

The four modules or \texttt{SpaDES}-deck can be called a global script
(available below). In this project, four modules are run:
\texttt{CBM\_defaults}, \texttt{CBM\_dataPrep\_RIA\_scenario},
\texttt{CBM\_vol2biomass\_RIA}, and \texttt{CBM\_core}. The code
environment is on a public repository. The interactions with other
models in our simulations are all via the
\texttt{CBM\_dataPrep\_scenario} modules. We simulate four scenarios and
therefore have four \texttt{CBM\_dataPrep\_scenario} modules described
in this appendix.

\hypertarget{background}{%
\section{Background}\label{background}}

This adaptation of the CBM was developed as a R\&D tool for improving
the science-basis for the CBM and applied carbon modelling in general.
This \texttt{SpaDES}-deck was developed on the \texttt{SpaDES} platform
(a package in R; \url{https://cran.r-project.org/package=SpaDES}) to
make it transparent, spatial explicit, and able to link to other
modules/models in \texttt{SpaDES}, striving towards a PERFICT approach.
This \texttt{SpaDES}-deck enables the inclusion of carbon modelling with
CBM in cumulative effects evaluation, and provides an environment in
which science improvements can be explored and tested. \texttt{SpaDES}
functions as a scheduler through space and time. Note that modules or
models do not have to be written in R, but callable from R. More
information on \texttt{SpaDES} and other openly available
\texttt{SpaDES} modules can be found here
\url{http://spades.predictiveecology.org/}.

\hypertarget{four-module-spades-deck}{%
\section{\texorpdfstring{Four-module
\texttt{SpaDES}-deck}{Four-module SpaDES-deck}}\label{four-module-spades-deck}}

By definition, a \texttt{SpaDES} environment is spatially explicit. In
addition to that modification, this \texttt{SpaDES}-deck differs
slightly from the CBM: the C transactions are calculated via matrix
multiplications as opposed to the non-matrix multiplications used in the
CBM. These multiplications are completed via a C++ script compiled in
\texttt{CBM\_core}. Knowledge of the \texttt{SpaDES} structure would
help an R-knowledgeable user to manipulate simulations but is not
necessary to run the current simulations. Prior knowledge of CBM-CFS3
would also help users understand the structure of these modules; the
default parameters used, but is not necessary to run simulations. The
simulations on which we base our research can be performed with the code
block presented below once all the \texttt{SpaDES} modules have been
cloned and access to all inputs confirmed. We describe all four modules
necessary for simulations performed in our study with parameters
described in Kurz et al.~(2009) and in Stinson et al.~(2011). Growth
curves as the main change-agent (m\^{}3/ha) for the study area are
available via URL in the \texttt{CBM\_vol2biomass\_RIA} module described
below. GitHub.

\hypertarget{cbm_defaults}{%
\subsection{\texorpdfstring{\texttt{CBM\_defaults}}{CBM\_defaults}}\label{cbm_defaults}}

This module loads all the CBM-CFS3 default parameters (Canadian defaults
that is akin to the Archive Index access database in CBM-CFS3). These
parameters are then stored in an S4 object
\texttt{RIAscenarioRuns\$cbmData} and accessed throughout the
simulations. This object has the following slot names:

• turnoverRates • rootParameters • decayParameters • spinupParameters •
classifierValues • climate • spatialUnitIds • slowAGtoBGTransferRate •
biomassToCarbonRate • ecoIndicess • puIndices • stumpParameters •
overmatureDeclineParameters • disturbanceMatrix.

Default parameters are stored in an SQLite database in this RStudio
project associated with the current simulations
(\texttt{spadesCBM\_RIA.Rproj} -
\emph{\textasciitilde spadesCBM\_RIA/data/modules/CBM\_defaults/data/cbm\_defaults}).
All parameters used in these simulations are the general Canadian
defaults, and can be accessed with common R functionality. In the
\texttt{SpaDES} environment, this module modules have events that are
scheduled for the simulation. This module has one event (\texttt{init})
and does not schedule anything else. It requires the R-object
\texttt{dbPath} and \texttt{sqlDir} to run, both specified in the global
script below.

\hypertarget{cbm_dataprep_studyarea_specifyscenario}{%
\subsection{\texorpdfstring{\texttt{CBM\_dataPrep\_studyArea\_specifyScenario}}{CBM\_dataPrep\_studyArea\_specifyScenario}}\label{cbm_dataprep_studyarea_specifyscenario}}

This module reads in user-provided information similarly to CBM-CFS3.
User provided/expected input include: • the ages of the stands/pixels
(in our case a raster), • study location information (in our case a
raster) • disturbance information the user wants applied in the
simulations (coma separated value (.csv) file) • the growth curves
(.csv) and where they should be applied (which pixels - raster) on the
land base, • growth curve meta data which includes at a minimum growth
curve identification and leading species from which a six column table
will be built in this module, OR the user can provide the six-column
meta data directly we have done for these simulations (.csv).

The \texttt{.inputObjects} section of this module provides links to all
input data for the simulations used in our study.

The user-provided study area is used to make a
\texttt{RIAscenarioRuns\$masterRaster} (see \texttt{.inputObjects}
section of this module) on which all maps and other calculations are
based. The spatial unit raster as well as an ecozone raster are created
using the \texttt{RIAscenarioRuns\$masterRaster}. Spatial units (SPUs)
are an overlay of administrative boundaries (provinces and territories)
and ecozones. SPUs are the link back to the default ecological
parameters assembled for CBM-CFS3 simulations in Canada. These
parameters are necessary to be able to perform a simulation; you either
use the defaults or have to provide alternative values for all the
parameters. The location information provided by SPU is used to narrow
the parameter options from \texttt{CBM\_default} to the ones that are
specific to this study area. The \texttt{CBM\_default} modules needs to
have been run before \texttt{CBM\_dataPrep\_studyArea\_scenario}.
Information and data provided are also used to create a table of similar
pixels to increase processing speeds (\texttt{pixelGroup}). A
\texttt{data.table} listing the \texttt{pixelGoup} in the initial
landscape is saved in the simList as \texttt{RIAscenarioRuns\$level3DT}.
All necessary vectors for annual processes (main part of the simulations
that happen in the \texttt{CBM\_core} module) are created in this
module. These vectors need to be in a specific format for the C++
functions processing. A table stored in the simList
(\texttt{RIAscenarioRuns\$mySpuDmid}) links the user-provided
disturbance information to the disturbance matrix identification numbers
in \texttt{CBM\_defaults}. The \texttt{.inputObjects} function at the
end of this module provides quasi-automatic read-in of all the necessary
rasters and tables for our simulations. This module is the most specific
to a study area. Users should expect this module to contain all
idiosyncratic data manipulations specific to the study area and to each
simulated scenario. Each scenario requires its own data preparation
modules. These are names as follows: \texttt{CBM\_dataPrep\_RIAfri},
\texttt{CBM\_dataPrep\_RIApresentDay},
\texttt{CBM\_dataPrep\_RIAharvest1}, and
\texttt{CBM\_dataPrep\_RIAharvest2}.

\hypertarget{cbm_vol2biomass_studyarea}{%
\subsection{\texorpdfstring{\texttt{CBM\_vol2biomass\_studyArea}}{CBM\_vol2biomass\_studyArea}}\label{cbm_vol2biomass_studyarea}}

This module is a translation module. It takes in the growth curves
(cumulative \(m3/ha\)) and returns the biomass increments that drive the
simulations. It is an implementation of the stand-level biomass
conversion parameters published in Boudewyn et al.~(2007). Similarly to
the \texttt{CBM\_dataPrep\_myStudyArea\_scenario} module, this module is
specific to our study area.

Following the CBM-CFS3 approach, the growth curves are cumulative
\(m3/ha\). Each curves needs to have an identification number permitting
the linking to its spatial application; it needs the range of ages from
0 to the oldest ages represented on the landscape; it needs the volume
associated with that age vector; and a leading species for that curve.
Metadata for each curve was already provided in the data preparation
module (\texttt{CBM\_dataPrep\_studyArea\_specifyScenario}). All
\(m3/ha\) curves provided are plotted for visual inspection in a simList
object named \texttt{RIAscenarioRuns\$volCurves}. The unaltered three
above ground carbon pools directly out of the application of the
Boudewyn et al.~(2007) parameters and caps, resulted in our in our study
as in most cases, in some non-smooth curves, curves with odd shapes, or
curves that to not go through a 0 intercept. For these reason, this
script incorporates a smoothing procedure that uses a Chapman-Richards
function to correct for non-plausible shapes and wiggles in the curves
resulting from the translation process. Note that the purpose of the
present SpaDES-deck is to emulate the CBM-CFS3 approach. This module
saves figures for users to evaluate in
\emph{\textasciitilde spadesCBM\_RIA/data/modules/CBM\_vol2biomass\_RIA/figures}.
The \texttt{CBM\_dataPrep\_yourStudyArea\_specifyScenario} need to be
run prior to running this module. This module, however, could be run
independently for translation of single-species stand-level \(m3/ha\)
into three increment columns for the aboveground live biomass pools.

\hypertarget{cbm_core}{%
\subsection{\texorpdfstring{\texttt{CBM\_core}}{CBM\_core}}\label{cbm_core}}

This module uses all the information from the three other
\texttt{SpaDES} modules to compute the C simulations. The
\texttt{CBM\_default}, \texttt{CBM\_vol2biomass\_studyArea} and
\texttt{CBM\_dataPrep\_studyArea\_specifyScenario} modules need to be
run prior to this module. This module has six \texttt{SpaDES}-events
that can be scheduled: \texttt{spinup}, \texttt{postSpinup},
\texttt{saveSpinup}, \texttt{annual}, \texttt{plot}, and
\texttt{savePools}. We do not schedule the \texttt{saveSpinup} event for
our simulations. The \texttt{spinup} event is the \texttt{init} event
run by default in \texttt{SpaDES} modules. This event call a C++ spinup
function. This emulated the the traditional initialization procedure of
the CBM where each stand (\texttt{pixelGroup} in our case) is disturbed
using the disturbance specified in
\texttt{RIAscenarioRuns\$historicDMIDs} (usually wildfire for the
ecozone) and re-grown using the provided above ground biomass pools,
repeatedly, until the dead organic matter (DOM) pools values stabilize
or the maximum number of iteration is reached
\texttt{RIAscenarioRuns\$maxRotations}. In this \texttt{spinup} event,
carbon increment estimates from the biomass estimate of Boudewyn et
al.~(2007)'s translation of the \(m3/ha\) curves from the
\texttt{CBM\_vol2biomass} module are used for each of the aboveground
live pools. The bark, branches, biomass nonmerch (equation 2 in
Boudewyn) pools, and biomass sap (equation 3 in Boudewyn) are grouped
under ``other'' in the CBM. Biomass in coarse and fine roots are
estimated using the above ground estimates from the increments and
default parameters, one set for softwood and one set for hardwood (see
\texttt{root\_parameter} table in the SQLite default database). To
estimate carbon in all other pools, the burn-grow cycle is repeated as
described above. In all spatial units in Canada, the historical
disturbance is set to fire. The \texttt{CBM\_default} module has fire
return intervals for each ecozone in Canada that can be match with the
ecozone of the study area via the ecozone raster, which is what we did
for our simulations. Once the \texttt{spinup} is complete, the last
cycle grows the pixelGroup (still using the same growth curve) to the
user-provided age in the age raster. In the \texttt{postSpinup} event,
matrices are set up for the processes that will happen in the
\texttt{annual} event. In order, the processes are: disturbance, half
growth, dead organic matter (DOM) turnover, biomass turnover, over
mature decline calculations, second half of growth, DOM decay, slow
decay, and slow soil mixing of dead pools' carbon. The \texttt{annual}
event is where all the processes are applied. Most carbon transactions
calculated via C++ functions compiled via Rcpp in this module.

The current global script produces a series of default plots. The
\texttt{plot} event uses three parameters: the initial plot time
(\texttt{.plotInitialTime}), the interval to plot
(\texttt{.plotInterval}), and the carbon pools to plot
(\texttt{poolsToPlot}). The parameter \texttt{poolsToPlot} accepts a
character vector consisting of any individual pools in the object
\texttt{RIAscenarioRuns\$cbmPools} as well as totalCarbon for the sum of
below ground and above ground carbon. The event \texttt{savePools} is
scheduled last. It creates a \texttt{.csv} file
(\texttt{cPoolsPixelYear.csv}) that contains the carbon pool values for
each \texttt{pixelGroup} at the end of each simulation year, for all
simulation years. The event \texttt{spinupDebug} is currently a place
holder to explore spinup results, it is set to FALSE in our simulations.

\hypertarget{current-simulations}{%
\section{Current Simulations}\label{current-simulations}}

The global script in this document will run by default, simulations for
\(53.4Mha\) of managed forests in the northeastern corner of the
province of British Colombia, Canada, running the following script, with
all the defaults parameters, for four scenarios. The first scenario
provides an estimate of the carbon carrying capacity (Liang et al.,
2017) of the study area by simulating the landscape using the CBM and
the fire model (Armstrong \& Cumming, 2003 - Appendix 3) from the year
2020 to 20540 (FRI for Fire Return Interval). The second represents the
present day landscape, where disturbances from 1985-2015 as presented in
Hermosilla et al.~(2016) were applied annually and resulting C dynamics
simulated with the CBM. The final two scenarios simulate two levels in
resource removal as simulated by ws3 (Paradis et al., 2018- Appendix 2)
, on representing a sustainable resource management extraction scenario
(base harvest) and the other a lesser amount of resources removed (less
harvest). Note that all scenarios use the same
\texttt{CBM\_vol2biomass\_RIA} module, and that both the
\texttt{CBM\_default} and \texttt{CBM\_core} modules are non-changing
between scenario. Meta data for each scenario are created in the script
below and this script runs four separate simulations.

\begin{Shaded}
\begin{Highlighting}[]
\FunctionTok{library}\NormalTok{(Require)}
\FunctionTok{Require}\NormalTok{(}\StringTok{"magrittr"}\NormalTok{) }\CommentTok{\# this is needed to use "\%\textgreater{}\%" below}
\FunctionTok{Require}\NormalTok{(}\StringTok{"SpaDES.core"}\NormalTok{)}

\FunctionTok{install\_github}\NormalTok{(}\StringTok{"PredictiveEcology/CBMutils@development"}\NormalTok{)}
\CommentTok{\#load\_all("\textasciitilde{}/GitHub/PredictiveEcology/CBMutils")}
\FunctionTok{Require}\NormalTok{(}\StringTok{"PredictiveEcology/CBMutils (\textgreater{}= 0.0.6)"}\NormalTok{)}

\FunctionTok{options}\NormalTok{(}\StringTok{"reproducible.useRequire"} \OtherTok{=} \ConstantTok{TRUE}\NormalTok{)}

\NormalTok{cacheDir }\OtherTok{\textless{}{-}}\NormalTok{ reproducible}\SpecialCharTok{::}\FunctionTok{checkPath}\NormalTok{(}\StringTok{"cache"}\NormalTok{, }\AttributeTok{create =} \ConstantTok{TRUE}\NormalTok{)}
\NormalTok{moduleDir }\OtherTok{\textless{}{-}}\NormalTok{ reproducible}\SpecialCharTok{::}\FunctionTok{checkPath}\NormalTok{(}\StringTok{"modules"}\NormalTok{)}
\NormalTok{inputDir }\OtherTok{\textless{}{-}}\NormalTok{ reproducible}\SpecialCharTok{::}\FunctionTok{checkPath}\NormalTok{(}\StringTok{"inputs"}\NormalTok{, }\AttributeTok{create =} \ConstantTok{TRUE}\NormalTok{)}
\NormalTok{outputDir }\OtherTok{\textless{}{-}}\NormalTok{ reproducible}\SpecialCharTok{::}\FunctionTok{checkPath}\NormalTok{(}\StringTok{"outputs"}\NormalTok{, }\AttributeTok{create =} \ConstantTok{TRUE}\NormalTok{)}
\NormalTok{scratchDir }\OtherTok{\textless{}{-}} \FunctionTok{file.path}\NormalTok{(}\FunctionTok{tempdir}\NormalTok{(), }\StringTok{"scratch"}\NormalTok{, }\StringTok{"CBM"}\NormalTok{) }\SpecialCharTok{\%\textgreater{}\%}\NormalTok{ reproducible}\SpecialCharTok{::}\FunctionTok{checkPath}\NormalTok{(}\AttributeTok{create =} \ConstantTok{TRUE}\NormalTok{)}

\DocumentationTok{\#\# }\AlertTok{TODO}\DocumentationTok{ fix this so we can run all the times with the appropriate sims below}
\NormalTok{timesFRI }\OtherTok{\textless{}{-}} \FunctionTok{list}\NormalTok{(}\AttributeTok{start =} \FloatTok{2020.00}\NormalTok{, }\AttributeTok{end =} \FloatTok{2540.00}\NormalTok{)}
\NormalTok{timesPresentDay }\OtherTok{\textless{}{-}} \FunctionTok{list}\NormalTok{(}\AttributeTok{start =} \FloatTok{1985.00}\NormalTok{, }\AttributeTok{end =} \FloatTok{2015.00}\NormalTok{)}
\NormalTok{timesHarvest1 }\OtherTok{\textless{}{-}} \FunctionTok{list}\NormalTok{(}\AttributeTok{start =} \FloatTok{2020.00}\NormalTok{, }\AttributeTok{end =} \FloatTok{2099.00}\NormalTok{)}


\NormalTok{parameters }\OtherTok{\textless{}{-}} \FunctionTok{list}\NormalTok{(}
  \AttributeTok{CBM\_defaults =} \FunctionTok{list}\NormalTok{(}
    \AttributeTok{.useCache =} \ConstantTok{TRUE}
\NormalTok{  ),}
  \AttributeTok{CBM\_vol2biomass\_RIA =} \FunctionTok{list}\NormalTok{(}
    \AttributeTok{.useCache =} \ConstantTok{TRUE}
\NormalTok{  )}
\NormalTok{)}
\CommentTok{\#Fire Return Interval}
\NormalTok{parametersFRI }\OtherTok{\textless{}{-}}\NormalTok{ parameters}
\NormalTok{parametersFRI}\SpecialCharTok{$}\NormalTok{CBM\_dataPrep\_RIAfri }\OtherTok{\textless{}{-}} \FunctionTok{list}\NormalTok{(}
                     \AttributeTok{.useCache =} \ConstantTok{TRUE}
\NormalTok{                     )}
\NormalTok{parametersFRI}\SpecialCharTok{$}\NormalTok{CBM\_core }\OtherTok{\textless{}{-}} \FunctionTok{list}\NormalTok{(}
                    \AttributeTok{.useCache =} \ConstantTok{FALSE}\NormalTok{, }\CommentTok{\#"init", \#c(".inputObjects", "init")}
                    \CommentTok{\# .plotInterval = 5,}
                    \AttributeTok{.plotInitialTime =}\NormalTok{ timesFRI}\SpecialCharTok{$}\NormalTok{start,}
                    \AttributeTok{poolsToPlot =} \FunctionTok{c}\NormalTok{(}\StringTok{"totalCarbon"}\NormalTok{),}
                    \AttributeTok{spinupDebug =} \ConstantTok{FALSE} \DocumentationTok{\#\# }\AlertTok{TODO}\DocumentationTok{: temporary}
\NormalTok{                  )}
\CommentTok{\#Present Day 1985{-}2015}
\NormalTok{parametersPresentDay }\OtherTok{\textless{}{-}}\NormalTok{ parameters}
\NormalTok{parametersPresentDay}\SpecialCharTok{$}\NormalTok{CBM\_dataPrep\_RIApresentDay }\OtherTok{\textless{}{-}} \FunctionTok{list}\NormalTok{(}
                    \AttributeTok{.useCache =} \ConstantTok{TRUE}
\NormalTok{                    )}
\NormalTok{parametersPresentDay}\SpecialCharTok{$}\NormalTok{CBM\_core }\OtherTok{\textless{}{-}} \FunctionTok{list}\NormalTok{(}
                    \AttributeTok{.useCache =} \ConstantTok{FALSE}\NormalTok{, }\CommentTok{\#"init", \#c(".inputObjects", "init")}
                    \CommentTok{\# .plotInterval = 5,}
                    \AttributeTok{.plotInitialTime =}\NormalTok{ timesPresentDay}\SpecialCharTok{$}\NormalTok{start,}
                    \AttributeTok{poolsToPlot =} \FunctionTok{c}\NormalTok{(}\StringTok{"totalCarbon"}\NormalTok{),}
                    \AttributeTok{spinupDebug =} \ConstantTok{FALSE} \DocumentationTok{\#\# }\AlertTok{TODO}\DocumentationTok{: temporary}
\NormalTok{                  )}
\CommentTok{\# harvest1 is the base case}
\NormalTok{parametersHarvest1 }\OtherTok{\textless{}{-}}\NormalTok{ parameters}
\NormalTok{parametersHarvest1}\SpecialCharTok{$}\NormalTok{CBM\_dataPrep\_RIAharvest1 }\OtherTok{\textless{}{-}} \FunctionTok{list}\NormalTok{(}
                    \AttributeTok{.useCache =} \ConstantTok{TRUE}
\NormalTok{                    )}

\NormalTok{parametersHarvest1}\SpecialCharTok{$}\NormalTok{CBM\_core }\OtherTok{\textless{}{-}} \FunctionTok{list}\NormalTok{(}
                    \CommentTok{\#.useCache = TRUE, \#"init", \#c(".inputObjects", "init")}
                    \CommentTok{\# .plotInterval = 5,}
                    \AttributeTok{.plotInitialTime =}\NormalTok{ timesHarvest1}\SpecialCharTok{$}\NormalTok{start,}
                    \AttributeTok{poolsToPlot =} \FunctionTok{c}\NormalTok{(}\StringTok{"totalCarbon"}\NormalTok{),}
                    \AttributeTok{spinupDebug =} \ConstantTok{FALSE}\NormalTok{) }\DocumentationTok{\#\# }\AlertTok{TODO}\DocumentationTok{: temporary}

\NormalTok{modulesFRI }\OtherTok{\textless{}{-}} \FunctionTok{list}\NormalTok{(}\StringTok{"CBM\_defaults"}\NormalTok{, }\StringTok{"CBM\_dataPrep\_RIAfri"}\NormalTok{, }\StringTok{"CBM\_vol2biomass\_RIA"}\NormalTok{, }\StringTok{"CBM\_core"}\NormalTok{)}
\NormalTok{modulesPresentDay }\OtherTok{\textless{}{-}} \FunctionTok{list}\NormalTok{(}\StringTok{"CBM\_defaults"}\NormalTok{, }\StringTok{"CBM\_dataPrep\_RIApresentDay"}\NormalTok{, }\StringTok{"CBM\_vol2biomass\_RIA"}\NormalTok{, }\StringTok{"CBM\_core"}\NormalTok{)}
\NormalTok{modulesHarvest1 }\OtherTok{\textless{}{-}} \FunctionTok{list}\NormalTok{(}\StringTok{"CBM\_defaults"}\NormalTok{, }\StringTok{"CBM\_dataPrep\_RIAharvest1"}\NormalTok{, }\StringTok{"CBM\_vol2biomass\_RIA"}\NormalTok{, }\StringTok{"CBM\_core"}\NormalTok{)}
\CommentTok{\#harvest2 is the less{-}than{-}base case}
\NormalTok{modulesHarvest2 }\OtherTok{\textless{}{-}} \FunctionTok{list}\NormalTok{(}\StringTok{"CBM\_defaults"}\NormalTok{, }\StringTok{"CBM\_dataPrep\_RIAharvest2"}\NormalTok{, }\StringTok{"CBM\_vol2biomass\_RIA"}\NormalTok{, }\StringTok{"CBM\_core"}\NormalTok{)}

\NormalTok{objects }\OtherTok{\textless{}{-}} \FunctionTok{list}\NormalTok{(}
  \AttributeTok{dbPath =} \FunctionTok{file.path}\NormalTok{(inputDir, }\StringTok{"cbm\_defaults"}\NormalTok{, }\StringTok{"cbm\_defaults.db"}\NormalTok{),}
  \AttributeTok{sqlDir =} \FunctionTok{file.path}\NormalTok{(inputDir, }\StringTok{"cbm\_defaults"}\NormalTok{)}
\NormalTok{)}

\NormalTok{paths }\OtherTok{\textless{}{-}} \FunctionTok{list}\NormalTok{(}
  \AttributeTok{cachePath =}\NormalTok{ cacheDir,}
  \AttributeTok{modulePath =}\NormalTok{ moduleDir,}
  \AttributeTok{inputPath =}\NormalTok{ inputDir,}
  \AttributeTok{rasterPath =}\NormalTok{ scratchDir}
\NormalTok{)}

\NormalTok{pathsFRI }\OtherTok{\textless{}{-}}\NormalTok{ paths}
\NormalTok{pathsFRI}\SpecialCharTok{$}\NormalTok{outputPath }\OtherTok{\textless{}{-}} \FunctionTok{file.path}\NormalTok{(outputDir,}\StringTok{"FRI"}\NormalTok{)}

\NormalTok{pathsPresentDay }\OtherTok{\textless{}{-}}\NormalTok{ paths}
\NormalTok{pathsPresentDay}\SpecialCharTok{$}\NormalTok{outputPath }\OtherTok{\textless{}{-}} \FunctionTok{file.path}\NormalTok{(outputDir,}\StringTok{"presentDay"}\NormalTok{)}

\NormalTok{pathsHarvest1 }\OtherTok{\textless{}{-}}\NormalTok{ paths}
\NormalTok{pathsHarvest1}\SpecialCharTok{$}\NormalTok{outputPath }\OtherTok{\textless{}{-}} \FunctionTok{file.path}\NormalTok{(outputDir,}\StringTok{"harvest1"}\NormalTok{)}

\NormalTok{pathsHarvest2 }\OtherTok{\textless{}{-}}\NormalTok{ paths}
\NormalTok{pathsHarvest2}\SpecialCharTok{$}\NormalTok{outputPath }\OtherTok{\textless{}{-}} \FunctionTok{file.path}\NormalTok{(outputDir,}\StringTok{"harvest2"}\NormalTok{)}


\NormalTok{quickPlot}\SpecialCharTok{::}\FunctionTok{dev.useRSGD}\NormalTok{(}\ConstantTok{FALSE}\NormalTok{)}
\FunctionTok{dev}\NormalTok{()}
\FunctionTok{clearPlot}\NormalTok{()}
\FunctionTok{options}\NormalTok{(}\AttributeTok{spades.moduleCodeChecks =} \ConstantTok{FALSE}\NormalTok{,}
        \AttributeTok{reproducible.useMemoise =} \ConstantTok{FALSE}\NormalTok{,}
        \AttributeTok{spades.recoveryMode =} \ConstantTok{FALSE}\NormalTok{)}
        \CommentTok{\#reproducible.useNewDigestAlgorithm = 2)}

\CommentTok{\# this is the whens for the 540 years of fire runs}
\NormalTok{whensFRI }\OtherTok{\textless{}{-}} \FunctionTok{sort}\NormalTok{(}\FunctionTok{c}\NormalTok{(timesFRI}\SpecialCharTok{$}\NormalTok{start, timesFRI}\SpecialCharTok{$}\NormalTok{start }\SpecialCharTok{+} \DecValTok{1}\SpecialCharTok{:}\DecValTok{4} \SpecialCharTok{*} \DecValTok{100}\NormalTok{, timesFRI}\SpecialCharTok{$}\NormalTok{end }\SpecialCharTok{{-}} \DecValTok{2}\SpecialCharTok{:}\DecValTok{0}\NormalTok{))}
\CommentTok{\# this is for the presentDay runs}
\NormalTok{whensPresentDay }\OtherTok{\textless{}{-}} \FunctionTok{sort}\NormalTok{(}\FunctionTok{c}\NormalTok{(timesPresentDay}\SpecialCharTok{$}\NormalTok{start, timesPresentDay}\SpecialCharTok{$}\NormalTok{start }\SpecialCharTok{+} \FunctionTok{c}\NormalTok{(}\DecValTok{5}\NormalTok{, }\DecValTok{10}\NormalTok{, }\DecValTok{15}\NormalTok{, }\DecValTok{20}\NormalTok{, }\DecValTok{25}\NormalTok{), }
\NormalTok{                          timesPresentDay}\SpecialCharTok{$}\NormalTok{end }\SpecialCharTok{{-}} \DecValTok{2}\SpecialCharTok{:}\DecValTok{0}\NormalTok{))}
\CommentTok{\# this is for the harvest runs}
\NormalTok{whensHarvest1 }\OtherTok{\textless{}{-}} \FunctionTok{sort}\NormalTok{(}\FunctionTok{c}\NormalTok{(timesHarvest1}\SpecialCharTok{$}\NormalTok{start, timesHarvest1}\SpecialCharTok{$}\NormalTok{start }\SpecialCharTok{+} \FunctionTok{c}\NormalTok{(}\DecValTok{10}\NormalTok{, }\DecValTok{30}\NormalTok{, }\DecValTok{50}\NormalTok{, }\DecValTok{60}\NormalTok{), }
\NormalTok{                          timesHarvest1}\SpecialCharTok{$}\NormalTok{end }\SpecialCharTok{{-}} \DecValTok{1}\SpecialCharTok{:}\DecValTok{0}\NormalTok{))}


\NormalTok{outputsFRI }\OtherTok{\textless{}{-}} \FunctionTok{as.data.frame}\NormalTok{(}\FunctionTok{expand.grid}\NormalTok{(}\AttributeTok{objectName =} \FunctionTok{c}\NormalTok{(}\StringTok{"cbmPools"}\NormalTok{, }\StringTok{"NPP"}\NormalTok{), }\AttributeTok{saveTime =}\NormalTok{ whensFRI))}
\NormalTok{outputsPresentDay }\OtherTok{\textless{}{-}} \FunctionTok{as.data.frame}\NormalTok{(}\FunctionTok{expand.grid}\NormalTok{(}\AttributeTok{objectName =} \FunctionTok{c}\NormalTok{(}\StringTok{"cbmPools"}\NormalTok{, }\StringTok{"NPP"}\NormalTok{), }\AttributeTok{saveTime =}\NormalTok{ whensPresentDay))}
\NormalTok{outputsHarvest1 }\OtherTok{\textless{}{-}} \FunctionTok{as.data.frame}\NormalTok{(}\FunctionTok{expand.grid}\NormalTok{(}\AttributeTok{objectName =} \FunctionTok{c}\NormalTok{(}\StringTok{"cbmPools"}\NormalTok{, }\StringTok{"NPP"}\NormalTok{), }\AttributeTok{saveTime =}\NormalTok{ whensHarvest1))}

\DocumentationTok{\#\# All simulation parameters are set{-}up above this line.}
\DocumentationTok{\#\# All simulations are set{-}up (simInit) and performed (spades) below this line}

\NormalTok{RIApresentDayRuns }\OtherTok{\textless{}{-}} \FunctionTok{simInitAndSpades}\NormalTok{(}\AttributeTok{times =}\NormalTok{ timesPresentDay,}
                               \AttributeTok{params =}\NormalTok{ parametersPresentDay,}
                               \AttributeTok{modules =}\NormalTok{ modulesPresentDay,}
                               \AttributeTok{objects =}\NormalTok{ objects,}
                               \AttributeTok{paths =}\NormalTok{ pathsPresentDay,}
                               \AttributeTok{outputs =}\NormalTok{ outputsPresentDay,}
                               \AttributeTok{loadOrder =} \FunctionTok{unlist}\NormalTok{(modulesPresentDay),}
                               \AttributeTok{debug =} \ConstantTok{TRUE}\NormalTok{)}

\NormalTok{RIAfriRuns }\OtherTok{\textless{}{-}} \FunctionTok{simInitAndSpades}\NormalTok{(}\AttributeTok{times =}\NormalTok{ timesFRI,}
                               \AttributeTok{params =}\NormalTok{ parametersFRI,}
                               \AttributeTok{modules =}\NormalTok{ modulesFRI,}
                               \AttributeTok{objects =}\NormalTok{ objects,}
                               \AttributeTok{paths =}\NormalTok{ pathsFRI,}
                               \AttributeTok{outputs =}\NormalTok{ outputsFRI,}
                               \AttributeTok{loadOrder =} \FunctionTok{unlist}\NormalTok{(modulesFRI),}
                               \AttributeTok{debug =} \ConstantTok{TRUE}\NormalTok{)}

\NormalTok{RIAharvest1Runs }\OtherTok{\textless{}{-}} \FunctionTok{simInitAndSpades}\NormalTok{(}\AttributeTok{times =}\NormalTok{ timesHarvest1,}
                               \AttributeTok{params =}\NormalTok{ parametersHarvest1,}
                               \AttributeTok{modules =}\NormalTok{ modulesHarvest1,}
                               \AttributeTok{objects =}\NormalTok{ objects,}
                               \AttributeTok{paths =}\NormalTok{ pathsHarvest1,}
                               \AttributeTok{outputs =}\NormalTok{ outputsHarvest1,}
                               \AttributeTok{loadOrder =} \FunctionTok{unlist}\NormalTok{(modulesHarvest1),}
                               \AttributeTok{debug =} \ConstantTok{TRUE}\NormalTok{)}

\NormalTok{RIAharvest2Runs }\OtherTok{\textless{}{-}} \FunctionTok{simInitAndSpades}\NormalTok{(}\AttributeTok{times =}\NormalTok{ timesHarvest1,}
                               \AttributeTok{params =}\NormalTok{ parametersHarvest1,}
                               \AttributeTok{modules =}\NormalTok{ modulesHarvest2,}
                               \AttributeTok{objects =}\NormalTok{ objects,}
                               \AttributeTok{paths =}\NormalTok{ pathsHarvest2,}
                               \AttributeTok{outputs =}\NormalTok{ outputsHarvest1,}
                               \AttributeTok{loadOrder =} \FunctionTok{unlist}\NormalTok{(modulesHarvest2),}
                               \AttributeTok{debug =} \ConstantTok{TRUE}\NormalTok{)}
\end{Highlighting}
\end{Shaded}

\hypertarget{cbmutils}{%
\section{\texorpdfstring{\texttt{CBMutils}}{CBMutils}}\label{cbmutils}}

In the spirit of PERFICT, we developed a series of utility functions, to
support simulations using this \texttt{SpaDES}-deck. This are in the
package \texttt{CBMutils} and are available on a public github
repository.

\hypertarget{references}{%
\section{References}\label{references}}

Armstrong, G. W., \& Cumming, S. G. (2003). Estimating the Cost of Land
Base Changes Due to Wildfire Using Shadow Prices. Forest Science, 49(5),
719--730. \url{https://doi.org/10.1093/forestscience/49.5.719} Boudewyn,
P., Song, X., Magnussen, S., \& Gillis, M. D. (2007). Model-based,
volume-to-biomass conversion for forested and vegetated land in Canada
(P. F. C. Canadian Forest Service, Trans.; ISSN 0830-0453 ISBN
978-0-662-46513-3; Information Report BC-X-411). Natural Resources
Canada. Hermosilla, T., Wulder, M. A., White, J. C., Coops, N. C.,
Hobart, G. W., \& Campbell, L. B. (2016). Mass data processing of time
series Landsat imagery: Pixels to data products for forest monitoring.
International Journal of Digital Earth, 1--20.
\url{https://doi.org/10.1080/17538947.2016.1187673} Kurz, W. A., Dymond,
C. C., White, T. M., Stinson, G., Shaw, C. H., Rampley, G. J., Smyth,
C., Simpson, B. N., Neilson, E. T., Trofymow, J. A., Metsaranta, J., \&
Apps, M. J. (2009). CBM-CFS3: A model of carbon-dynamics in forestry and
land-use change implementing IPCC standards. Ecological Modelling,
220(4), 480--504. Liang, S., Hurteau, M. D., \& Westerling, A. L.
(2017). Potential decline in carbon carrying capacity under projected
climate-wildfire interactions in the Sierra Nevada. Scientific Reports,
7(1), 2420. \url{https://doi.org/10.1038/s41598-017-02686-0} McIntire,
E. J. B., Chubaty, A., Luo, Y., \& Cumming, S. G. (2019). Develop and
Run Spatially Explicit Discrete Event Simulation Models (2.07) {[}R
language{]}. \url{https://spades.predictiveecology.org},
\url{https://github.com/PredictiveEcology/SpaDES} Paradis, G., Bouchard,
M., LeBel, L., \& D'Amours, S. (2018). A bi-level model formulation for
the distributed wood supply planning problem. Canadian Journal of Forest
Research, 48(2), 160--171. \url{https://doi.org/10.1139/cjfr-2017-0240}
Stinson, G., Kurz, W. A., Smyth, C. E., Neilson, E. T., Dymond, C. C.,
Metsaranta, J. M., Boisvenue, C., Rampley, G. J., Li, Q., White, T. M.,
\& Blain, D. (2011). An inventory-based analysis of Canada's managed
forest carbon dynamics, 1990 to 2008. Global Change Biology, 17(6),
2227--2244. \url{https://doi.org/10.1111/j.1365-2486.2010.02369.x}

\end{document}
